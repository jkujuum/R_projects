\documentclass[]{article}
\usepackage{lmodern}
\usepackage{amssymb,amsmath}
\usepackage{ifxetex,ifluatex}
\usepackage{fixltx2e} % provides \textsubscript
\ifnum 0\ifxetex 1\fi\ifluatex 1\fi=0 % if pdftex
  \usepackage[T1]{fontenc}
  \usepackage[utf8]{inputenc}
\else % if luatex or xelatex
  \ifxetex
    \usepackage{mathspec}
  \else
    \usepackage{fontspec}
  \fi
  \defaultfontfeatures{Ligatures=TeX,Scale=MatchLowercase}
\fi
% use upquote if available, for straight quotes in verbatim environments
\IfFileExists{upquote.sty}{\usepackage{upquote}}{}
% use microtype if available
\IfFileExists{microtype.sty}{%
\usepackage{microtype}
\UseMicrotypeSet[protrusion]{basicmath} % disable protrusion for tt fonts
}{}
\usepackage[margin=1in]{geometry}
\usepackage{hyperref}
\hypersetup{unicode=true,
            pdftitle={Predicting readmission probability for diabetes inpatients},
            pdfauthor={STAT 471/571/701 Modern Data Mining},
            pdfborder={0 0 0},
            breaklinks=true}
\urlstyle{same}  % don't use monospace font for urls
\usepackage{color}
\usepackage{fancyvrb}
\newcommand{\VerbBar}{|}
\newcommand{\VERB}{\Verb[commandchars=\\\{\}]}
\DefineVerbatimEnvironment{Highlighting}{Verbatim}{commandchars=\\\{\}}
% Add ',fontsize=\small' for more characters per line
\usepackage{framed}
\definecolor{shadecolor}{RGB}{248,248,248}
\newenvironment{Shaded}{\begin{snugshade}}{\end{snugshade}}
\newcommand{\KeywordTok}[1]{\textcolor[rgb]{0.13,0.29,0.53}{\textbf{#1}}}
\newcommand{\DataTypeTok}[1]{\textcolor[rgb]{0.13,0.29,0.53}{#1}}
\newcommand{\DecValTok}[1]{\textcolor[rgb]{0.00,0.00,0.81}{#1}}
\newcommand{\BaseNTok}[1]{\textcolor[rgb]{0.00,0.00,0.81}{#1}}
\newcommand{\FloatTok}[1]{\textcolor[rgb]{0.00,0.00,0.81}{#1}}
\newcommand{\ConstantTok}[1]{\textcolor[rgb]{0.00,0.00,0.00}{#1}}
\newcommand{\CharTok}[1]{\textcolor[rgb]{0.31,0.60,0.02}{#1}}
\newcommand{\SpecialCharTok}[1]{\textcolor[rgb]{0.00,0.00,0.00}{#1}}
\newcommand{\StringTok}[1]{\textcolor[rgb]{0.31,0.60,0.02}{#1}}
\newcommand{\VerbatimStringTok}[1]{\textcolor[rgb]{0.31,0.60,0.02}{#1}}
\newcommand{\SpecialStringTok}[1]{\textcolor[rgb]{0.31,0.60,0.02}{#1}}
\newcommand{\ImportTok}[1]{#1}
\newcommand{\CommentTok}[1]{\textcolor[rgb]{0.56,0.35,0.01}{\textit{#1}}}
\newcommand{\DocumentationTok}[1]{\textcolor[rgb]{0.56,0.35,0.01}{\textbf{\textit{#1}}}}
\newcommand{\AnnotationTok}[1]{\textcolor[rgb]{0.56,0.35,0.01}{\textbf{\textit{#1}}}}
\newcommand{\CommentVarTok}[1]{\textcolor[rgb]{0.56,0.35,0.01}{\textbf{\textit{#1}}}}
\newcommand{\OtherTok}[1]{\textcolor[rgb]{0.56,0.35,0.01}{#1}}
\newcommand{\FunctionTok}[1]{\textcolor[rgb]{0.00,0.00,0.00}{#1}}
\newcommand{\VariableTok}[1]{\textcolor[rgb]{0.00,0.00,0.00}{#1}}
\newcommand{\ControlFlowTok}[1]{\textcolor[rgb]{0.13,0.29,0.53}{\textbf{#1}}}
\newcommand{\OperatorTok}[1]{\textcolor[rgb]{0.81,0.36,0.00}{\textbf{#1}}}
\newcommand{\BuiltInTok}[1]{#1}
\newcommand{\ExtensionTok}[1]{#1}
\newcommand{\PreprocessorTok}[1]{\textcolor[rgb]{0.56,0.35,0.01}{\textit{#1}}}
\newcommand{\AttributeTok}[1]{\textcolor[rgb]{0.77,0.63,0.00}{#1}}
\newcommand{\RegionMarkerTok}[1]{#1}
\newcommand{\InformationTok}[1]{\textcolor[rgb]{0.56,0.35,0.01}{\textbf{\textit{#1}}}}
\newcommand{\WarningTok}[1]{\textcolor[rgb]{0.56,0.35,0.01}{\textbf{\textit{#1}}}}
\newcommand{\AlertTok}[1]{\textcolor[rgb]{0.94,0.16,0.16}{#1}}
\newcommand{\ErrorTok}[1]{\textcolor[rgb]{0.64,0.00,0.00}{\textbf{#1}}}
\newcommand{\NormalTok}[1]{#1}
\usepackage{graphicx,grffile}
\makeatletter
\def\maxwidth{\ifdim\Gin@nat@width>\linewidth\linewidth\else\Gin@nat@width\fi}
\def\maxheight{\ifdim\Gin@nat@height>\textheight\textheight\else\Gin@nat@height\fi}
\makeatother
% Scale images if necessary, so that they will not overflow the page
% margins by default, and it is still possible to overwrite the defaults
% using explicit options in \includegraphics[width, height, ...]{}
\setkeys{Gin}{width=\maxwidth,height=\maxheight,keepaspectratio}
\IfFileExists{parskip.sty}{%
\usepackage{parskip}
}{% else
\setlength{\parindent}{0pt}
\setlength{\parskip}{6pt plus 2pt minus 1pt}
}
\setlength{\emergencystretch}{3em}  % prevent overfull lines
\providecommand{\tightlist}{%
  \setlength{\itemsep}{0pt}\setlength{\parskip}{0pt}}
\setcounter{secnumdepth}{0}
% Redefines (sub)paragraphs to behave more like sections
\ifx\paragraph\undefined\else
\let\oldparagraph\paragraph
\renewcommand{\paragraph}[1]{\oldparagraph{#1}\mbox{}}
\fi
\ifx\subparagraph\undefined\else
\let\oldsubparagraph\subparagraph
\renewcommand{\subparagraph}[1]{\oldsubparagraph{#1}\mbox{}}
\fi

%%% Use protect on footnotes to avoid problems with footnotes in titles
\let\rmarkdownfootnote\footnote%
\def\footnote{\protect\rmarkdownfootnote}

%%% Change title format to be more compact
\usepackage{titling}

% Create subtitle command for use in maketitle
\newcommand{\subtitle}[1]{
  \posttitle{
    \begin{center}\large#1\end{center}
    }
}

\setlength{\droptitle}{-2em}

  \title{Predicting readmission probability for diabetes inpatients}
    \pretitle{\vspace{\droptitle}\centering\huge}
  \posttitle{\par}
    \author{STAT 471/571/701 Modern Data Mining}
    \preauthor{\centering\large\emph}
  \postauthor{\par}
      \predate{\centering\large\emph}
  \postdate{\par}
    \date{Due midnight, April 7th, 2019}


\begin{document}
\maketitle

{
\setcounter{tocdepth}{4}
\tableofcontents
}
\section{Executive Summary}\label{executive-summary}

\subsection{Background}\label{background}

Hospital readmission is very expensive and it is an emerging indicator
of quality of health care. On average, Medicare spends double the amount
for an episode with one readmission. Moreover, diabetes is a chronic
medical condition affecting millions of Americans, but if managed well,
with good diet, exercise and medication, patients can lead relatively
normal lives. However, if improperly managed, diabetes can lead to
patients being continuously admitted and readmitted to hospitals.
Readmissions are especially serious - they represent a failure of the
health system to provide adequate support to the patient and are
extremely costly to the system. As a result, the Centers for Medicare
and Medicaid Services announced in 2012 that they would no longer
reimburse hospitals for services rendered if a patient was readmitted
with complications within 30 days of discharge.

Given these policy changes, being able to identify and predict those
patients most at risk for costly readmissions has become a pressing
priority for hospital administrators. In this project, we shall explore
how to use the techniques we have learned in order to help better manage
diabetes patients who have been admitted to a hospital. Our goal is to
avoid patients being readmitted within 30 days of discharge, which
reduces costs for the hospital and improves outcomes for patients. If we
could identify important factors relating to the chance of a patient
being readmitted within 30 days of discharge, effective intervention
could be done to reduce the chance of being readmitted. Also if we could
predict one's chance being readmitted well, actions can be taken.

\subsection{The data}\label{the-data}

The original data is from the
\href{https://archive.ics.uci.edu/ml/datasets/Diabetes+130-US+hospitals+for+years+1999-2008}{Center
for Clinical and Translational Research} at Virginia Commonwealth
University. It covers data on diabetes patients across 130 U.S.
hospitals from 1999 to 2008. There are over 100,000 unique hospital
admissions in this dataset, from \textasciitilde{}70,000 unique
patients. The data includes demographic elements, such as age, gender,
and race, as well as clinical attributes such as tests conducted,
emergency/inpatient visits, etc. Refer to the original documentation for
more details on the dataset. Three former students Spencer Luster,
Matthew Lesser and Mridul Ganesh, brought this data set into the class
and did a wonderful final project. We will use a subset processed by the
group but with a somewhat different objective.

All observations have five things in common:

\begin{enumerate}
\def\labelenumi{\arabic{enumi}.}
\tightlist
\item
  They are all hospital admissions
\item
  Each patient had some form of diabetes
\item
  The patient stayed for between 1 and 14 days.
\item
  The patient had laboratory tests performed on him/her.
\item
  The patient was given some form of medication during the visit.
\end{enumerate}

The data was collected during a ten-year period from 1999 to 2008. There
are over 100,000 unique hospital admissions in the data set, with
\textasciitilde{}70,000 unique patients.

\subsection{Methods used}\label{methods-used}

In order to analyse data, I first cleaned the data and deleted
unecessary variables.Then I label readmission into ``1'' as ``Yes,
readmitted within 30 days'' and ``0'' as ``No''. Then, we did
exploratory data analysis to have an overview of the data.

The data is seperated into two sets. One set is used and trained to
generate models, while the other set of data is used to test the model
and see if the model fit the data well. In order to find posible factors
that are related to readmission, I used two method to build 2 models.
The first model is Lasso model, and the second one is backward selection
model.We weighted the cost of mislabel a readmission to be twice as much
as mislabel a non-readmission We compared the two models based on their
prediction error and ROC curve.

The final model contains 7 variables: 1. number\_inpatient (Number of
inpatient visits by the patient in the year prior to the current
encounter) 2. number\_diagnoses (Total no. of diagnosis entered for the
patient) 3. insulin 4. diabetesMed (Indicates if any diabetes medication
was prescribed) 5. disch\_disp\_modified 6. diag1\_mod (ICD9 codes for
the primary diagnoses of the patient.) 7. diag3\_mod (ICD9 codes for the
tertiary diagnoses of the patient.)

\subsection{Limitation}\label{limitation}

Readmission is not only related to factors inside of the hospital. In
our model, we only related the readmission rate to patient
himself/herself or factors in hospital. However, whether or not a
patient is readmitted may be very related to factors outside of the
hospital. Do patients receive good care after leaving the hospital; did
they taking medicine on time; and their excercise frequency and
emotional mood can also be a potential factors to patients readmission
rate.

\section{Data Analysis}\label{data-analysis}

\subsection{(i) Data Summary}\label{i-data-summary}

\subsubsection{Description of variables}\label{description-of-variables}

The dataset used covers \textasciitilde{}50 different variables to
describe every hospital diabetes admission. In this section we give an
overview and brief description of the variables in this dataset.

\textbf{1) Patient identifiers:}

\begin{enumerate}
\def\labelenumi{\alph{enumi}.}
\tightlist
\item
  \texttt{encounter\_id}: unique identifier for each admission
\item
  \texttt{patient\_nbr}: unique identifier for each patient
\end{enumerate}

\textbf{2) Patient Demographics:}

\texttt{race}, \texttt{age}, \texttt{gender}, \texttt{weight} cover the
basic demographic information associated with each patient.
\texttt{Payer\_code} is an additional variable that identifies which
health insurance (Medicare /Medicaid / Commercial) the patient holds.

\textbf{3) Admission and discharge details:}

\begin{enumerate}
\def\labelenumi{\alph{enumi}.}
\tightlist
\item
  \texttt{admission\_source\_id} and \texttt{admission\_type\_id}
  identify who referred the patient to the hospital (e.g.~physician
  vs.~emergency dept.) and what type of admission this was (Emergency
  vs.~Elective vs.~Urgent).
\item
  \texttt{discharge\_disposition\_id} indicates where the patient was
  discharged to after treatment.
\end{enumerate}

\textbf{4) Patient Medical History:}

\begin{enumerate}
\def\labelenumi{\alph{enumi}.}
\tightlist
\item
  \texttt{num\_outpatient}: number of outpatient visits by the patient
  in the year prior to the current encounter
\item
  \texttt{num\_inpatient}: number of inpatient visits by the patient in
  the year prior to the current encounter
\item
  \texttt{num\_emergency}: number of emergency visits by the patient in
  the year prior to the current encounter
\end{enumerate}

\textbf{5) Patient admission details:}

\begin{enumerate}
\def\labelenumi{\alph{enumi}.}
\tightlist
\item
  \texttt{medical\_specialty}: the specialty of the physician admitting
  the patient
\item
  \texttt{diag\_1}, \texttt{diag\_2}, \texttt{diag\_3}: ICD9 codes for
  the primary, secondary and tertiary diagnoses of the patient. ICD9 are
  the universal codes that all physicians use to record diagnoses. There
  are various easy to use tools to lookup what individual codes mean
  (Wikipedia is pretty decent on its own)
\item
  \texttt{time\_in\_hospital}: the patient’s length of stay in the
  hospital (in days)
\item
  \texttt{number\_diagnoses}: Total no. of diagnosis entered for the
  patient
\item
  \texttt{num\_lab\_procedures}: No. of lab procedures performed in the
  current encounter
\item
  \texttt{num\_procedures}: No. of non-lab procedures performed in the
  current encounter
\item
  \texttt{num\_medications}: No. of distinct medications prescribed in
  the current encounter
\end{enumerate}

\textbf{6) Clinical Results:}

\begin{enumerate}
\def\labelenumi{\alph{enumi}.}
\tightlist
\item
  \texttt{max\_glu\_serum}: indicates results of the glucose serum test
\item
  \texttt{A1Cresult}: indicates results of the A1c test
\end{enumerate}

\textbf{7) Medication Details:}

\begin{enumerate}
\def\labelenumi{\alph{enumi}.}
\tightlist
\item
  \texttt{diabetesMed}: indicates if any diabetes medication was
  prescribed
\item
  \texttt{change}: indicates if there was a change in diabetes
  medication
\item
  \texttt{24\ medication\ variables}: indicate whether the dosage of the
  medicines was changed in any manner during the encounter
\end{enumerate}

\textbf{8) Readmission indicator:}

Indicates whether a patient was readmitted after a particular admission.
There are 3 levels for this variable: ``NO'' = no readmission,
``\textless{} 30'' = readmission within 30 days and ``\textgreater{}
30'' = readmission after more than 30 days. The 30 day distinction is of
practical importance to hospitals because federal regulations penalize
hospitals for an excessive proportion of such readmissions.

\subsubsection{EDA}\label{eda}

In our dataset, we have 101766 observations. It covers data on diabetes
patients across 130 U.S. hospitals from 1999 to 2008.We have 31
variables but encouter\_id and patient\_nbr are unrelevent to our model,
so we exclude those two. Readmitted as our output variable, we modified
it into ``1'': patient readmit within 30 days, ``0'', patient does not
readmit within 30 days.

From Bar plot in Appendix I, we can see that, more females are
readmitted than male, but the probability of readmission is about the
same (0.1266991 for female and 0.1243728 for male).

Among all race, Caucasians and African American are the most likely to
readmit, and more Caucasians are readmitted.

There are more 60-79 yrs old people being readmitted. As age goes up,
the ratio of readmission goes up.

In our dataset, we find 2273 ``?'' in race, here, we consider ``?'' as
mixed race, and we will take ``?'' into consideration. We also find 3
Unknown/Invalid sex in gender column. We will also take this 3
observations into consideration as representation of transgenders.

Name shown below will be our input, and \texttt{readmitted} will be our
response variable.

\begin{verbatim}
##  [1] "race"                "gender"              "time_in_hospital"   
##  [4] "num_lab_procedures"  "num_procedures"      "num_medications"    
##  [7] "number_outpatient"   "number_emergency"    "number_inpatient"   
## [10] "number_diagnoses"    "max_glu_serum"       "A1Cresult"          
## [13] "metformin"           "glimepiride"         "glipizide"          
## [16] "glyburide"           "pioglitazone"        "rosiglitazone"      
## [19] "insulin"             "change"              "diabetesMed"        
## [22] "disch_disp_modified" "adm_src_mod"         "adm_typ_mod"        
## [25] "age_mod"             "diag1_mod"           "diag2_mod"          
## [28] "diag3_mod"           "readmitted"
\end{verbatim}

\subsection{ii) Analyses}\label{ii-analyses}

We seperate the data set into 2 subsets. 70\% of the data is assigned
into training set and the remaining 30\% is assigned to testing data
set. There are 71236 variables in training set, and 30530 variables in
testing set.

\subsubsection{Lasso Model}\label{lasso-model}

The first model we built is a logistic regression with Lasso method. We
used cv.glmnet function, and binomial deviance from 10-fold cross
validation. We get the following graph. Then we choosed variables of
coef.1se, and fit those variables into a logistic regression.

glm(readmitted\textasciitilde{} time\_in\_hospital + number\_inpatient +
number\_diagnoses + insulin + diabetesMed + diabetesMed +
disch\_disp\_modified + diag1\_mod + diag3\_mod ,family=binomial,
data=data.train)

After eliminate variables, and make sure every variables are
statistically significant at 0.05 level, we get our lasso model -
fit.lasso with 7 variables

\includegraphics{Mini_Project_Diabetes_files/figure-latex/unnamed-chunk-8-1.pdf}

\subsubsection{Backward Selection}\label{backward-selection}

The second method we used to build the model is backward selection.
Started with the full model, kept one variable with that largest p-value
until each variable is significant at 0.05 level. We get our backward
selection model - fit.back with 15 variables.

\subsubsection{Model Comparision}\label{model-comparision}

We use Bayes Rule to choose a threshold for out model. Based on a quick
and somewhat arbitrary guess, we estimate it costs twice as much to
mislabel a readmission than it does to mislabel a non-readmission.

\(a_{01}/a_{10}=1/10=0.5\)

\(P(Y=1 \vert x) > \frac{\frac{a_{0,1}}{a_{1,0}}}{1 + \frac{a_{0,1}}{a_{1,0}}}\)

\(P(Y=1 \vert x) > \frac{0.5}{1 + 0.5}=0.33\)

Based on calculation, we get out threshold set to be 0.33. After
calculation, fit.lasso has a mce of 0.1165084, and fit.back has a mce of
0.1169014. We choose the model with lower mce. Thus, we choose lasso
model.

The following graph shows the ROC curve and AUC value for both fit.lasso
and fit.back. Two ROC curve looks very similar and we have AUC = 0.65
for both model. So, we still use the lasso model because it has a lower
prediction error.
\includegraphics{Mini_Project_Diabetes_files/figure-latex/unnamed-chunk-17-1.pdf}

\begin{verbatim}
##   AUC.fit.lasso. AUC.fit.back.
## 1           0.65          0.65
\end{verbatim}

\subsection{iii) Conclusion}\label{iii-conclusion}

In conclusion, our final model contains 7 variables.

number\_inpatient (Number of inpatient visits by the patient in the year
prior to the current encounter): As other variables remain the same,
increase in one unit of number\_inpatient will result in increase of
readmission by 0.258085. The hospital should limit the inpatient visits
to reduce the readmission probability.

number\_diagnoses (Total no. of diagnosis entered for the patient): As
other variables remain the same, increase in one unit of
number\_diagnoses will result in increase of readmission by 0.027515.
The more disgnosis entered for the patient, the patient is more likely
to readmit.

insulin: All insulin level has a negative effect on readmission. Insulin
is good for diabetes, so the negative relationship makes sense.

diabetesMed (Indicates if any diabetes medication was prescribed):
diabetesMed Yes has a higher positive effect on readmission

disch\_disp\_modified, All level has positive relationship with
readmission.

diag1\_mod (ICD9 codes for the primary diagnoses of the patient.): All
levels of diag1\_mod have positive relationship with readmission

diag3\_mod (ICD9 codes for the tertiary diagnoses of the patient.): Most
diad3\_mod level have positive relationship with readmission

Beside these variables, I suggest hospitals to pay more attention to
patient that are 60 yrs old and above. Based on our bar plot from EDA,
we find the probability increases as people getting older, which also
makes sense that older people are more vulnerable. We can also have some
further analysis on factors outside of the hospital to see if we can
find some valuable information.

\subsection{iii) Appendix}\label{iii-appendix}

\subsubsection{Readmission vs Gender}\label{readmission-vs-gender}

\includegraphics{Mini_Project_Diabetes_files/figure-latex/unnamed-chunk-20-1.pdf}

Readmission rate vs gender

\begin{verbatim}
## # A tibble: 3 x 2
##   gender          ratio
##   <fct>           <dbl>
## 1 Female          0.127
## 2 Male            0.124
## 3 Unknown/Invalid 0
\end{verbatim}

\subsubsection{Readmission vs Race}\label{readmission-vs-race}

\includegraphics{Mini_Project_Diabetes_files/figure-latex/unnamed-chunk-22-1.pdf}

\includegraphics{Mini_Project_Diabetes_files/figure-latex/unnamed-chunk-23-1.pdf}

\subsubsection{Readmission vs Age}\label{readmission-vs-age}

\includegraphics{Mini_Project_Diabetes_files/figure-latex/unnamed-chunk-24-1.pdf}

\begin{Shaded}
\begin{Highlighting}[]
\NormalTok{ratio_age <-}\StringTok{ }\NormalTok{readmission }\OperatorTok\StringTok{ }\KeywordTok{group_by}\NormalTok{(age_mod) }\OperatorTok\StringTok{ }\KeywordTok{summarise}\NormalTok{(}\DataTypeTok{ratio=}\KeywordTok{sum}\NormalTok{(readmitted}\OperatorTok{==}\StringTok{"1"}\NormalTok{)}\OperatorTok{/}\KeywordTok{sum}\NormalTok{(readmitted}\OperatorTok{==}\StringTok{"0"}\NormalTok{)) }
\NormalTok{ratio_age }\OperatorTok\StringTok{ }\KeywordTok{ggplot}\NormalTok{(}\KeywordTok{aes}\NormalTok{(}\DataTypeTok{x =}\NormalTok{ ratio_age}\OperatorTok{$}\NormalTok{age_mod, }\DataTypeTok{y =}\NormalTok{ ratio_age}\OperatorTok{$}\NormalTok{ratio)) }\OperatorTok{+}\StringTok{ }
\StringTok{   }\KeywordTok{geom_point}\NormalTok{(}\DataTypeTok{size=}\DecValTok{5}\NormalTok{) }\OperatorTok{+}\StringTok{ }
\StringTok{   }\KeywordTok{labs}\NormalTok{(}\DataTypeTok{x=}\StringTok{"Age"}\NormalTok{,}\DataTypeTok{y=}\StringTok{"Readmission Ratio"}\NormalTok{, }\DataTypeTok{title=}\StringTok{"Readmission Ratio vs Age"}\NormalTok{)}
\end{Highlighting}
\end{Shaded}

\includegraphics{Mini_Project_Diabetes_files/figure-latex/unnamed-chunk-25-1.pdf}

\section{Collaboration}\label{collaboration}

This is an \textbf{individual} assignment. We will only allow private
Piazza posts for questions. If there are questions that are generally
useful, we will release that information.


\end{document}
